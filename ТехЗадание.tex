\section{Техническое задание}

\subsection{Основание для разработки}

Основанием для разработки является задание на преддипломную практику по направлению подготовки 09.03.04 «Программная инженерия», а также тема выпускной квалификационной работы: «Программно-информационная система для проведения школьных предметных олимпиад».

\subsection{Цель разработки}

Цель разработки --- создание программно-информационной системы, обеспечивающей автоматизацию ключевых процессов организации и проведения школьных предметных олимпиад, включая регистрацию участников, проверку работ, публикацию результатов и обработку апелляций.

\subsection{Назначение системы}

Разрабатываемая система предназначена для:
\begin{itemize}
\item ведения единого реестра олимпиадных мероприятий;
\item сопровождения взаимодействия организаторов, образовательных организаций, жюри и участников;
\item учета и хранения результатов олимпиад и материалов участников;
\item снижения трудоёмкости организационных процедур;
\item повышения прозрачности и контролируемости процессов.
\end{itemize}

\subsection{Объект автоматизации}

Объектом автоматизации являются организационные, учетные и информационные процессы школьного этапа предметных олимпиад:
\begin{itemize}
\item планирование и запуск олимпиад;
\item формирование и подтверждение состава участников;
\item прием работ и передача на экспертную проверку;
\item фиксация результатов и формирование протоколов;
\item работа с апелляциями и публикация итогов.
\end{itemize}

\subsection{Характеристика пользователей системы}

Эксплуатация системы предполагает участие нескольких категорий пользователей:
\begin{itemize}
\item администраторы, обеспечивающие сопровождение и настройку;
\item организаторы, управляющие олимпиадными мероприятиями;
\item представители образовательных организаций;
\item члены жюри;
\item участники олимпиад.
\end{itemize}

Для каждой роли определяется свой набор разрешённых операций и видимость данных.

\subsection{Функциональные требования}

Система должна обеспечивать следующие функциональные возможности:
\begin{enumerate}
\item \textbf{Управление пользователями и доступом:}
\begin{itemize}
\item регистрация и авторизация пользователей;
\item разграничение прав доступа по ролям;
\item поддержка смены и восстановления учетных данных.
\end{itemize}

\item \textbf{Управление олимпиадами:}
\begin{itemize}
\item создание и редактирование карточек олимпиад;
\item задание предмета, сроков и параметров проведения;
\item мониторинг статусов проведения.
\end{itemize}

\item \textbf{Работа с участниками:}
\begin{itemize}
\item регистрация и учет участников;
\item хранение сопроводительных данных;
\item контроль корректности заявочных сведений.
\end{itemize}

\item \textbf{Проверка и публикация результатов:}
\begin{itemize}
\item передача работ на проверку жюри;
\item фиксация баллов и комментариев;
\item формирование и публикация итоговых результатов.
\end{itemize}

\item \textbf{Апелляционные процедуры:}
\begin{itemize}
\item прием обращений участников;
\item регистрация решений по апелляциям;
\item отражение изменений в итоговых результатах.
\end{itemize}

\item \textbf{Информационное сопровождение:}
\begin{itemize}
\item публикация новостей и объявлений;
\item поддержка обратной связи;
\item формирование оперативной отчетности.
\end{itemize}
\end{enumerate}

\subsection{Требования к информационному обеспечению}

Информационное обеспечение строится на использовании реляционной базы данных. Укрупненный состав информационных сущностей представлен в таблице~\ref{tab:entities}.

\begin{xltabular}{\textwidth}{|l|X|}
\caption{Основные информационные сущности системы\label{tab:entities}}\\ \hline
\centrow Сущность & \centrow Назначение \\ \hline
\thead{1} & \thead{2} \\ \hline
\endfirsthead
\continuecaption{Продолжение таблицы \ref{tab:entities}}
\thead{1} & \thead{2} \\ \hline
\finishhead
Пользователи & Хранение учетных записей, ролей и параметров доступа \\ \hline
Олимпиады & Хранение параметров проведения и статусов олимпиад \\ \hline
Участники & Учет данных участников и их заявок \\ \hline
Работы участников & Хранение материалов, передаваемых на проверку \\ \hline
Результаты и апелляции & Фиксация баллов, решений жюри и итоговых изменений \\ \hline
Служебные сообщения и новости & Информационное сопровождение и коммуникации пользователей
\end{xltabular}

\subsection{Требования к программно-техническим средствам}

Для эксплуатации системы требуется:
\begin{itemize}
\item серверная платформа с поддержкой PHP 8.x;
\item система управления базами данных PostgreSQL;
\item веб-сервер для публикации приложения;
\item клиентский веб-браузер с поддержкой современных веб-стандартов.
\end{itemize}

\subsection{Требования к надёжности и безопасности}

Система должна обеспечивать:
\begin{itemize}
\item устойчивость к типовым ошибкам пользовательского ввода;
\item контроль целостности хранимых данных;
\item разграничение прав доступа к критическим операциям;
\item защиту персональных данных в соответствии с нормативными требованиями;
\item журналирование значимых действий пользователей;
\item возможность резервного копирования и восстановления данных.
\end{itemize}

\subsection{Требования к интерфейсу}

Интерфейс системы должен соответствовать следующим требованиям:
\begin{itemize}
\item единообразие навигации и терминологии для всех ролей;
\item доступность основных операций не более чем за 2--3 шага;
\item информативные сообщения о статусах выполнения операций;
\item адаптация страниц к использованию на типовых разрешениях экранов.
\end{itemize}

\subsection{Показатели качества системы}

Для оценки качества реализации принимаются следующие показатели:
\begin{itemize}
\item полнота покрытия функциональных требований;
\item среднее время отклика интерфейса на пользовательские действия;
\item доля успешно завершённых сценариев без ошибок ввода;
\item корректность формирования итоговых протоколов;
\item непротиворечивость данных после выполнения массовых операций.
\end{itemize}

\subsection{Состав и содержание работ}

В период преддипломной практики предусматривается выполнение работ:
\begin{enumerate}
\item анализ предметной области и нормативных требований;
\item формирование и согласование требований к системе;
\item подготовка проектных решений по архитектуре и структуре данных;
\item оформление отчёта о практике с иллюстративными материалами.
\end{enumerate}

\subsection{Критерии приемки результатов практики}

Результаты считаются достигнутыми при выполнении следующих условий:
\begin{itemize}
\item сформирован и оформлен полный комплект текстовых разделов отчёта;
\item подготовлены проектные решения, согласованные с задачами ВКР;
\item обеспечена логическая целостность представления материалов;
\item отражены требования к безопасности, надёжности и сопровождению системы.
\end{itemize}
