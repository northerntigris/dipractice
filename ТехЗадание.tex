\section{Техническое задание}

\subsection{Основание для разработки}

Основанием для выполнения работ является задание на преддипломную практику по направлению подготовки 09.03.04 «Программная инженерия». Объект разработки --- информационная система «Единая цифровая платформа школьных олимпиад» (УДПСО), размещённая в каталоге \texttt{udpso}.

\subsection{Цель и назначение разработки}

Целью разработки является повышение эффективности организации школьного этапа олимпиад за счёт перехода к единой цифровой платформе, обеспечивающей централизованное хранение данных, прозрачные процедуры взаимодействия и снижение объёма ручных операций.

Назначение системы:
\begin{itemize}
\item автоматизация организационных и учётных процессов проведения олимпиад;
\item поддержка взаимодействия между организаторами, школами, жюри и участниками;
\item обеспечение публикации результатов и рассмотрения апелляций;
\item формирование статистики и оперативной отчётности.
\end{itemize}

\subsection{Требования к функциональным характеристикам}

Система должна обеспечивать реализацию следующих функций:
\begin{itemize}
\item регистрация, авторизация и восстановление доступа пользователей;
\item разграничение прав доступа по ролям (администратор, организатор, школа, жюри, ученик);
\item создание и редактирование олимпиад, управление календарём и статусами;
\item назначение членов жюри и контроль публикации результатов;
\item подача заявок, ведение списков участников, загрузка файлов работ;
\item выставление баллов, обработка апелляций, публикация итогов;
\item ведение новостного раздела и обращений в поддержку.
\end{itemize}

\subsection{Требования к информационному обеспечению}

В системе используется реляционная база данных PostgreSQL. Основные сущности предметной области и их назначение представлены в таблице~\ref{tab:entities}.

\begin{xltabular}{\textwidth}{|l|l|X|}
\caption{Ключевые сущности базы данных УДПСО\label{tab:entities}}\\ \hline
\centrow Сущность & \centrow Назначение & \centrow Основные атрибуты \\ \hline
\thead{1} & \thead{2} & \thead{3} \\ \hline
\endfirsthead
\continuecaption{Продолжение таблицы \ref{tab:entities}}
\thead{1} & \thead{2} & \thead{3} \\ \hline
\finishhead
\texttt{users} & Пользователи системы & ФИО, e-mail, логин, пароль (хэш), роль, связь с организацией \\ \hline
\texttt{olympiads} & Олимпиады организатора & Предмет, даты проведения, классы, статус, описание \\ \hline
\texttt{school\_olympiads} & Олимпиады школы & Привязка к школе и выбранной олимпиаде, локальные параметры \\ \hline
\texttt{olympiad\_participants} & Участники олимпиад & Данные участника, заявка, результат, статус апелляции \\ \hline
\texttt{olympiad\_jury} & Состав жюри & Идентификаторы олимпиады и члена жюри, роль эксперта \\ \hline
\texttt{participant\_work\_files} & Файлы работ & Имя файла, MIME-тип, двоичные данные, время загрузки \\ \hline
\texttt{news} & Новости платформы & Заголовок, текст, автор, дата публикации \\ \hline
\texttt{support\_chats} и \texttt{support\_messages} & Поддержка пользователей & Идентификатор чата, сообщения, статус прочтения, время отправки
\end{xltabular}

\subsection{Требования к программному обеспечению}

Для функционирования системы требуется следующий программный стек:
\begin{itemize}
\item веб-сервер Apache/Nginx с поддержкой PHP~8.x;
\item PostgreSQL 14+;
\item клиентская часть в современном браузере с поддержкой ES6;
\item ОС семейства Linux/Windows для развёртывания серверной части.
\end{itemize}

\subsection{Требования к надёжности и безопасности}

Система должна обеспечивать:
\begin{itemize}
\item хранение паролей в виде криптографических хэшей;
\item проверку сессии и прав доступа для защищённых API;
\item защиту персональных данных в соответствии с требованиями законодательства;
\item ведение журнала действий (создание олимпиад, публикация результатов, изменение состава жюри);
\item сохранность пользовательских файлов и возможность резервного копирования базы данных.
\end{itemize}

\subsection{Этапы выполнения работ в период практики}

В период преддипломной практики выполнены следующие этапы:
\begin{enumerate}
\item анализ предметной области и нормативной базы;
\item изучение существующей структуры проекта УДПСО;
\item формализация технических требований к системе;
\item подготовка технического проекта с описанием архитектуры и алгоритмов;
\item формирование отчёта о преддипломной практике в формате XeLaTeX по шаблону ГОСТ.
\end{enumerate}
