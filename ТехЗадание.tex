\section{Техническое задание}

\subsection{Основание для разработки}

Полное наименование системы: «Программно-информационная система для проведения школьных предметных олимпиад».
Основанием для разработки программы является приказ ректора ЮЗГУ от «XX» апреля 2023 г. №XXXX-с «Об утверждении тем выпускных квалификационных работ»

\subsection{Назначение разработки}

Программно-информационная система предназначена для автоматизации 
процессов подготовки, организации, проведения и сопровождения 
школьных предметных олимпиад на муниципальном уровне. Разработка 
направлена на создание единой цифровой среды, обеспечивающей 
взаимодействие органов управления образованием, образовательных 
организаций и иных участников олимпиадного процесса.

Система обеспечивает централизованное хранение и обработку данных 
об образовательных организациях, проводимых олимпиадах, участниках, 
членах жюри и результатах оценивания. Реализация программного решения 
позволяет формализовать процедуры проведения олимпиад, повысить 
оперативность обмена информацией и обеспечить прозрачность процессов 
регистрации и подведения итогов.

В рамках системы реализуется ролевая модель разграничения доступа, 
обеспечивающая выполнение пользователями только тех функций, которые 
соответствуют их полномочиям. В системе предусмотрены следующие 
категории пользователей:

\begin{itemize}
	\item системный администратор, осуществляющий техническое 
	администрирование, настройку параметров системы и контроль 
	корректности ее функционирования;
	
	\item модератор системы, обеспечивающий контроль размещаемой 
	информации и соблюдение регламентов работы;
	
	\item уполномоченный представитель органа местного самоуправления, 
	осуществляющего управление в сфере образования, выполняющий функции 
	организатора школьных предметных олимпиад на муниципальном уровне;
	
	\item представитель образовательной организации, осуществляющий 
	взаимодействие с организатором олимпиад и обеспечивающий работу 
	образовательной организации в системе;
	
	\item школьный координатор, ответственный за регистрацию участников, 
	сопровождение их участия и контроль корректности представляемых данных;
	
	\item член жюри, осуществляющий проверку и оценивание олимпиадных работ 
	с последующим внесением результатов в систему;
	
	\item участник олимпиады, проходящий регистрацию и принимающий участие 
	в соответствующих олимпиадных мероприятиях.
\end{itemize}

Назначение разработки заключается в обеспечении автоматизированной 
поддержки полного цикла проведения школьных предметных олимпиад: 
формирования перечня мероприятий, регистрации участников, назначения 
членов жюри, внесения и обработки результатов оценивания, а также 
формирования итоговой отчетной документации. Система позволяет 
оперативно получать сводные данные, формировать рейтинговые списки 
и протоколы, обеспечивать хранение архивной информации.

Использование программно-информационной системы способствует 
снижению трудоемкости организационных процессов, уменьшению 
вероятности ошибок, связанных с ручной обработкой данных, а также 
повышению уровня информационной открытости и управляемости 
процесса проведения школьных предметных олимпиад.

Таким образом, разработка направлена на создание эффективного 
инструмента цифровой поддержки муниципального этапа олимпиадного 
движения, обеспечивающего надежность хранения данных, структурированность 
информации и удобство взаимодействия всех категорий пользователей.

\subsection{Требования к программной системе}

\subsubsection{Требования к данным программно-информационной системы}

Программно-информационная система должна обеспечивать хранение, 
обработку и структурирование данных, необходимых для организации 
и проведения школьных предметных олимпиад.

Данные системы должны храниться в централизованной базе данных 
с обеспечением их целостности, непротиворечивости и актуальности. 
Структура базы данных должна предусматривать логическое разделение 
информации в соответствии с функциональными задачами системы.

В системе должны храниться следующие категории данных:

– сведения об образовательных организациях (наименование, 
идентификационные данные, контактная информация);

– сведения о пользователях системы (фамилия, имя, отчество, 
контактные данные, логин, роль в системе);

– сведения о школьных предметных олимпиадах (наименование, 
предметная область, сроки проведения, статус мероприятия);

– сведения об участниках олимпиад (персональные данные, 
образовательная организация, класс обучения);

– сведения о членах жюри;

– данные о регистрации участников на конкретные олимпиады;

– сведения о результатах оценивания олимпиадных работ;

– данные о сформированных протоколах и итоговых документах.

Персональные данные пользователей должны храниться в соответствии 
с требованиями законодательства Российской Федерации в области 
защиты персональных данных. Система должна обеспечивать разграничение 
доступа к данным в зависимости от роли пользователя.

Данные, вводимые в систему, должны проходить проверку на корректность 
и полноту. Не допускается сохранение записей, нарушающих логическую 
целостность базы данных. Для обеспечения надежности хранения 
информации должна быть предусмотрена возможность резервного 
копирования базы данных.

Выходными данными системы являются сформированные протоколы, 
рейтинговые списки, сводные отчеты, а также сведения о результатах 
участия в олимпиадах, доступные пользователям в соответствии 
с их правами доступа.
