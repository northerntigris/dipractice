\section{Техническое задание}

\subsection{Основание для разработки}

Основанием для разработки является задание на преддипломную практику по направлению 09.03.04 «Программная инженерия» и тема выпускной квалификационной работы: «Программно-информационная система для проведения школьных предметных олимпиад».

\subsection{Цель и назначение системы}

Цель разработки --- создание единой цифровой среды, обеспечивающей автоматизацию подготовки, проведения и подведения итогов школьных предметных олимпиад.

Назначение системы:
\begin{itemize}
\item централизованное ведение данных об олимпиадах, участниках и результатах;
\item поддержка взаимодействия организаторов, образовательных организаций, жюри и обучающихся;
\item сокращение сроков обработки заявок, работ и апелляций;
\item повышение прозрачности и контролируемости процедур.
\end{itemize}

\subsection{Функциональные требования}

Система должна обеспечивать:
\begin{itemize}
\item регистрацию и авторизацию пользователей;
\item ролевое разграничение прав доступа;
\item формирование и сопровождение олимпиад по предметам и классам;
\item учёт участников и загрузку файлов работ;
\item проверку работ, выставление баллов и публикацию результатов;
\item подачу и рассмотрение апелляций;
\item формирование статистической и оперативной отчётности.
\end{itemize}

\subsection{Требования к информационному обеспечению}

Хранение данных осуществляется в реляционной базе данных PostgreSQL. Ключевые сущности представлены в таблице~\ref{tab:entities}.

\begin{xltabular}{\textwidth}{|l|X|}
\caption{Ключевые сущности информационной системы\label{tab:entities}}\\ \hline
\centrow Сущность & \centrow Назначение \\ \hline
\thead{1} & \thead{2} \\ \hline
\endfirsthead
\continuecaption{Продолжение таблицы \ref{tab:entities}}
\thead{1} & \thead{2} \\ \hline
\finishhead
Пользователи & Хранение учётных записей, ролей и связей с образовательными организациями \\ \hline
Олимпиады & Учёт параметров проведения, предметов, сроков и статусов \\ \hline
Участники олимпиад & Регистрация школьников, заявок, результатов и апелляций \\ \hline
Состав жюри & Назначение экспертов и разграничение полномочий проверки \\ \hline
Файлы работ & Хранение материалов участников и сопроводительных документов \\ \hline
Новости и сообщения поддержки & Информационное сопровождение и коммуникации пользователей
\end{xltabular}

\subsection{Требования к программно-технической среде}

Для эксплуатации системы необходимы:
\begin{itemize}
\item веб-сервер с поддержкой PHP 8.x;
\item СУБД PostgreSQL;
\item современный веб-браузер на стороне пользователя;
\item локальная или серверная инфраструктура, обеспечивающая сетевой доступ к приложению.
\end{itemize}

\subsection{Требования к безопасности и надёжности}

Система должна обеспечивать:
\begin{itemize}
\item безопасное хранение аутентификационных данных;
\item проверку прав доступа для критичных операций;
\item защиту персональных данных в соответствии с действующим законодательством;
\item журналирование значимых действий пользователей;
\item возможность резервного копирования и восстановления данных.
\end{itemize}

\subsection{Этапы работ}

В рамках преддипломной практики выполняются:
\begin{enumerate}
\item анализ предметной области и нормативной документации;
\item формирование требований к программно-информационной системе;
\item разработка проектных решений по архитектуре и структуре данных;
\item подготовка отчёта о преддипломной практике в формате XeLaTeX.
\end{enumerate}
