\section{Анализ предметной области}

\subsection{Нормативно-правовое регулирование организации школьного этапа олимпиад}

Организация и проведение школьного этапа олимпиад в Российской Федерации осуществляется в рамках действующего законодательства в сфере образования. Правовые основы функционирования системы образования закреплены в Федеральном законе №~273-ФЗ «Об образовании в Российской Федерации». Данный нормативный акт определяет право обучающихся на участие в интеллектуальных и творческих конкурсах, а также устанавливает полномочия органов государственной власти и органов местного самоуправления в сфере образования.

Порядок организации и проведения всероссийской олимпиады школьников регламентируется приказом Минпросвещения России №~678 «Об утверждении Порядка проведения всероссийской олимпиады школьников». Документ определяет структуру олимпиады, последовательность этапов, требования к формированию оргкомитетов и жюри, правила проверки работ и порядок рассмотрения апелляций.

При обработке персональных данных участников и членов жюри применяются положения Федерального закона №~152-ФЗ «О персональных данных». Система должна обеспечивать разграничение прав доступа, хранение персональных сведений в защищённой базе данных и возможность журналирования действий пользователей.

Таким образом, предметная область имеет выраженную нормативную регламентацию, что требует формализации процедур и минимизации операций, выполняемых вручную.

\subsection{Характеристика объекта автоматизации}

Объектом автоматизации является процесс организации и проведения школьного этапа олимпиад в общеобразовательных организациях. В рамках текущего проекта автоматизируются следующие ключевые блоки:
\begin{itemize}
\item регистрация организаторов и школ на платформе;
\item управление пользователями и ролевой моделью доступа;
\item создание олимпиад и назначение параметров проведения;
\item подача заявок и формирование списков участников;
\item загрузка работ участников и проверка жюри;
\item публикация результатов и обработка апелляций;
\item формирование оперативной статистики и новостной ленты.
\end{itemize}

В неавтоматизированной модели эти операции распределены между таблицами, почтой и бумажными формами, что приводит к дублированию данных, росту количества ошибок и увеличению времени подготовки отчётности.

\subsection{Анализ участников процесса}

В процессе проведения школьного этапа олимпиад участвуют несколько категорий пользователей:
\begin{itemize}
\item \textbf{Администратор платформы} --- управляет пользователями, модерирует контент, контролирует состояние системы;
\item \textbf{Организатор} --- формирует олимпиады, назначает жюри, контролирует публикацию результатов;
\item \textbf{Школа и школьный координатор} --- подают заявки участников, ведут учёт обучающихся и сопровождают участие;
\item \textbf{Жюри} --- проверяет работы, выставляет баллы и рассматривает апелляции;
\item \textbf{Ученик} --- участвует в олимпиадах, подаёт работу, знакомится с результатом и при необходимости инициирует апелляцию.
\end{itemize}

Ролевое разделение напрямую отражено в интерфейсах проекта: реализованы отдельные кабинеты \texttt{dashboard-admin.html}, \texttt{dashboard-organizer.html}, \texttt{dashboard-school.html}, \texttt{dashboard-jury.html}, \texttt{dashboard-student.html}, что обеспечивает адаптацию набора функций под обязанности конкретного участника процесса.

\subsection{Анализ существующей реализации проекта УДПСО}

Исходный проект в каталоге \texttt{udpso} представляет собой веб-приложение с клиентской частью на HTML/CSS/JavaScript и серверной частью на PHP, взаимодействующей с СУБД PostgreSQL.

Клиентская часть реализует:
\begin{itemize}
\item публичные страницы платформы (главная, новости, архив, описание);
\item модальные формы авторизации и регистрации;
\item личные кабинеты для пяти ролей;
\item модули управления олимпиадами, участниками, пользователями и обращениями.
\end{itemize}

Серверная часть построена в формате набора REST-подобных конечных точек (папка \texttt{udpso/api}). API покрывает полный жизненный цикл олимпиады: от создания и назначения жюри до публикации результатов, загрузки файлов работ и обработки апелляций.

База данных \texttt{udpsodb.sql} содержит основные сущности: \texttt{users}, \texttt{olympiads}, \texttt{school\_olympiads}, \texttt{olympiad\_participants}, \texttt{olympiad\_jury}, \texttt{participant\_work\_files}, \texttt{support\_chats}, \texttt{support\_messages}, \texttt{news} и другие. Такая структура поддерживает хранение нормативно значимых данных и обеспечивает прослеживаемость действий пользователей.

\subsection{Проблемы и направления совершенствования}

В ходе изучения проекта и предметной области выделены направления дальнейшего развития:
\begin{itemize}
\item усиление валидации входных данных на уровне API;
\item расширение механизма аудита операций пользователей;
\item стандартизация обработки ошибок для клиентских интерфейсов;
\item добавление автоматических тестов для критичных бизнес-процессов;
\item внедрение резервного копирования и регламентов восстановления данных.
\end{itemize}

С учётом выявленных особенностей внедрение и развитие единой цифровой платформы школьных олимпиад является обоснованным и позволяет обеспечить прозрачность процедур, снижение трудоёмкости и повышение качества управления олимпиадным движением.
