\section{Анализ предметной области}
\subsection{Нормативно-правовое регулирование организации школьного этапа олимпиад}

Организация и проведение школьного этапа олимпиад в Российской Федерации осуществляется в рамках действующего законодательства в сфере образования. Правовые основы функционирования системы образования закреплены в Федеральный закон № 273-ФЗ «Об образовании в Российской Федерации». Указанный нормативный акт определяет права обучающихся на участие в интеллектуальных и творческих конкурсах, а также устанавливает полномочия органов государственной власти и органов местного самоуправления в сфере образования.

Порядок организации и проведения всероссийской олимпиады школьников регламентируется Приказ Минпросвещения России № 678 «Об утверждении Порядка проведения всероссийской олимпиады школьников». Документ определяет структуру олимпиады, последовательность этапов, требования к формированию оргкомитетов и жюри, правила проверки работ и порядок рассмотрения апелляций.

В соответствии с указанным Порядком организатором школьного этапа олимпиады являются органы местного самоуправления, осуществляющие управление в сфере образования. К их компетенции относится утверждение сроков проведения, формирование состава жюри, обеспечение организационного сопровождения и контроль соблюдения установленного регламента.

Кроме того, при обработке персональных данных участников подлежат применению положения Федеральный закон № 152-ФЗ «О персональных данных», устанавливающего требования к сбору, хранению и защите персональной информации.

Таким образом, предметная область характеризуется высокой степенью нормативной регламентации и требует строгого соблюдения установленных процедур.
\subsection{Характеристика объекта автоматизациия}

Объектом автоматизации в рамках данной работы является процесс организации и проведения школьного этапа олимпиад в общеобразовательных организациях.

Школьный этап представляет собой начальный уровень олимпиады, целью которого является выявление обучающихся, проявляющих устойчивый интерес и способности к изучению отдельных учебных дисциплин. Результаты школьного этапа используются для формирования состава участников муниципального этапа.

Процесс проведения включает следующие функциональные блоки:
\begin{itemize}
\item планирование и утверждение сроков проведения;
\item формирование перечня предметов;
\item регистрация образовательных организаций и обучающихся;
\item проведение олимпиады;
\item проверка работ и формирование протоколов;
\item публикация результатов;
\item рассмотрение апелляций;
\item формирование отчетности для органов управления образованием.
\end{itemize}