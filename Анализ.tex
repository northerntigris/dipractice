\section{Анализ предметной области}

\subsection{Нормативно-правовое регулирование проведения школьных предметных олимпиад}

Проведение школьного этапа предметных олимпиад относится к образовательной деятельности, регламентируемой действующим законодательством Российской Федерации. Базовые правовые основания определяются Федеральным законом №~273-ФЗ «Об образовании в Российской Федерации», который закрепляет права обучающихся на участие в конкурсных и олимпиадных мероприятиях, а также определяет полномочия образовательных организаций и органов управления образованием.

Порядок организации олимпиад, требования к процедурам проведения, проверке работ и подведению итогов определяются приказом Минпросвещения России №~678 «Об утверждении Порядка проведения всероссийской олимпиады школьников». Практическое значение данного документа заключается в формализации этапности процесса, регламента взаимодействия оргкомитета и жюри, а также правил информирования участников.

Поскольку при проведении олимпиад обрабатываются персональные данные обучающихся и сотрудников образовательных организаций, обязательному соблюдению подлежат требования Федерального закона №~152-ФЗ «О персональных данных». Это означает необходимость реализации технических и организационных мер защиты: разграничения прав доступа, ограничения перечня обрабатываемых данных, контроля действий пользователей и обеспечения целостности хранимой информации.

\subsection{Организационная модель предметной области}

Предметная область включает совокупность участников, документов и процедур, возникающих при планировании, проведении и завершении олимпиадного цикла. На практике процесс является многоэтапным и требует координации между образовательными организациями, организаторами и экспертным сообществом.

Ключевые этапы организационной модели:
\begin{enumerate}
\item формирование календарного плана олимпиад и параметров проведения;
\item формирование состава участников и подтверждение заявок;
\item проведение олимпиадных испытаний;
\item прием и хранение работ участников;
\item проверка работ и формирование первичных результатов;
\item рассмотрение апелляций;
\item публикация итоговых результатов и формирование отчётности.
\end{enumerate}

\begin{figure}[H]
\centering
\fbox{\parbox[c][0.25\textheight][c]{0.93\linewidth}{\centering\textbf{ОРГАНИЗАЦИОННАЯ СХЕМА ПРОВЕДЕНИЯ ШКОЛЬНОЙ ОЛИМПИАДЫ}\\[1ex]
\textit{МЕСТО ДЛЯ ВСТАВКИ ИЗОБРАЖЕНИЯ}}}
\caption{Общая организационная модель проведения олимпиады}
\label{fig:org-model}
\end{figure}

\subsection{Анализ участников и ролей}

В предметной области выделяются следующие категории участников:
\begin{itemize}
\item \textbf{администратор платформы} --- обеспечивает сопровождение системы, контроль корректности данных, управление учетными записями и правами доступа;
\item \textbf{организатор} --- отвечает за формирование олимпиад, управление сроками и публикацию итоговых результатов;
\item \textbf{представитель образовательной организации} --- регистрирует участников, сопровождает подачу документов и взаимодействует с организатором;
\item \textbf{член жюри} --- выполняет проверку работ, выставление баллов и участие в рассмотрении апелляций;
\item \textbf{участник олимпиады} --- проходит регистрацию, выполняет задания, загружает работу, знакомится с результатами.
\end{itemize}

Взаимодействие между ролями предполагает распределение ответственности, что делает ролевой механизм доступа обязательным требованием к программно-информационной системе.

\subsection{Проблемы традиционного подхода}

До внедрения единой системы процессы часто реализуются с использованием разрозненных инструментов (электронные таблицы, файлообмен, бумажные протоколы), что формирует ряд системных проблем:
\begin{itemize}
\item дублирование данных и отсутствие единого источника достоверной информации;
\item повышенная вероятность ошибок ручного ввода;
\item длительные сроки согласования и публикации результатов;
\item сложность контроля статусов выполнения процедур;
\item затруднения при формировании сводной отчётности.
\end{itemize}

Следовательно, автоматизация должна быть направлена не только на цифровизацию документов, но и на построение целостного управляемого бизнес-процесса.

\subsection{Требования к автоматизации с позиции предметной области}

На основании анализа предметной области формируются ключевые требования к программно-информационной системе:
\begin{itemize}
\item централизованное хранение данных об олимпиадах, участниках, работах и результатах;
\item поддержка полного жизненного цикла олимпиады в рамках единого интерфейса;
\item механизм ролевого доступа и журналирование критичных операций;
\item обеспечение прозрачности процедур проверки и апелляции;
\item возможность формирования оперативной и итоговой отчётности.
\end{itemize}

\begin{figure}[H]
\centering
\fbox{\parbox[c][0.25\textheight][c]{0.93\linewidth}{\centering\textbf{СХЕМА ВЗАИМОДЕЙСТВИЯ РОЛЕЙ В СИСТЕМЕ}\\[1ex]
\textit{МЕСТО ДЛЯ ВСТАВКИ ИЗОБРАЖЕНИЯ}}}
\caption{Взаимодействие участников процесса в программно-информационной системе}
\label{fig:roles-interaction}
\end{figure}

\subsection{Выводы по разделу}

Анализ предметной области показал, что проведение школьных предметных олимпиад является нормативно регламентированным и организационно сложным процессом, требующим надежного цифрового инструмента сопровождения. Разработка программно-информационной системы обеспечивает переход к управляемой модели проведения олимпиад, снижает долю ручных операций и повышает качество принятия организационных решений.
