\section{Анализ предметной области}

\subsection{Нормативно-правовое регулирование проведения школьных предметных олимпиад}

Организация и проведение школьного этапа олимпиад в Российской Федерации осуществляется в рамках действующего законодательства в сфере образования. Базовым документом является Федеральный закон №~273-ФЗ «Об образовании в Российской Федерации», закрепляющий право обучающихся на участие в интеллектуальных состязаниях и определяющий полномочия участников образовательных отношений.

Требования к порядку проведения олимпиад определяются приказом Минпросвещения России №~678 «Об утверждении Порядка проведения всероссийской олимпиады школьников». Документ задаёт структуру этапов, правила формирования оргкомитетов и жюри, порядок проверки работ, публикации результатов и рассмотрения апелляций.

При работе с персональными данными участников, организаторов и членов жюри подлежат соблюдению требования Федерального закона №~152-ФЗ «О персональных данных». Это накладывает ограничения на хранение, передачу и обработку информации и требует реализации механизмов разграничения прав доступа.

\subsection{Характеристика объекта автоматизации}

Объектом автоматизации является процесс подготовки и проведения школьных предметных олимпиад. Для рассматриваемой предметной области характерны большое количество участников, строгое соблюдение сроков и значительный объём данных, обрабатываемых в течение коротких временных интервалов.

Ключевые функциональные блоки процесса:
\begin{itemize}
\item планирование олимпиад по предметам и классам;
\item регистрация образовательных организаций и участников;
\item организационное сопровождение проведения;
\item проверка работ и фиксация результатов;
\item публикация итогов и работа с апелляциями;
\item формирование отчётности для органов управления образованием.
\end{itemize}

\subsection{Анализ участников процесса}

В рамках предметной области выделяются следующие группы участников:
\begin{itemize}
\item администраторы, выполняющие управление платформой и контроль целостности данных;
\item организаторы, отвечающие за проведение олимпиад и выпуск итоговых протоколов;
\item представители образовательных организаций, сопровождающие участие школьников;
\item жюри, осуществляющее проверку работ и рассмотрение апелляций;
\item обучающиеся, принимающие участие в олимпиадах и взаимодействующие с результатами.
\end{itemize}

Информационный обмен между участниками должен быть оперативным, достоверным и фиксируемым. Следовательно, необходима единая программно-информационная система с ролевой моделью доступа и централизованным хранением данных.

\subsection{Анализ текущего состояния программной реализации}

Исследуемая реализация представляет собой веб-ориентированную систему с клиентской и серверной частями, работающими с единой реляционной базой данных. Клиентская часть обеспечивает интерфейсы для основных ролей и операции жизненного цикла олимпиады, серверная --- предоставляет прикладной программный интерфейс для авторизации, управления олимпиадами, участниками, результатами и обращениями.

По результатам анализа установлено, что структура системы в целом соответствует потребностям предметной области и покрывает базовые сценарии проведения школьных предметных олимпиад: от планирования до публикации итогов.

\begin{figure}[H]
\centering
\fbox{\parbox[c][0.24\textheight][c]{0.92\linewidth}{\centering\textbf{СХЕМА ПРЕДМЕТНОЙ ОБЛАСТИ И РОЛЕЙ ПОЛЬЗОВАТЕЛЕЙ}\\[1ex]
\textit{МЕСТО ДЛЯ ВСТАВКИ ИЗОБРАЖЕНИЯ}}}
\caption{Структура предметной области проведения школьных предметных олимпиад}
\label{fig:domain-structure}
\end{figure}

\subsection{Проблемы и обоснование необходимости дальнейшего развития}

Несмотря на наличие базовой реализации, остаются направления, требующие развития:
\begin{itemize}
\item унификация валидации входных данных и форматов ошибок;
\item расширение средств аудита действий пользователей;
\item повышение отказоустойчивости за счёт регламентов резервного копирования;
\item усиление методического сопровождения эксплуатации системы на стороне образовательных организаций.
\end{itemize}

Таким образом, разработка по теме «Программно-информационная система для проведения школьных предметных олимпиад» является актуальной и практически значимой, а выбранное направление автоматизации полностью соответствует задачам преддипломной практики.
