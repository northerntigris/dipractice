\section{Технический проект}

\subsection{Общая архитектура системы}

Система УДПСО реализована по клиент-серверной архитектуре. На клиентской стороне используются HTML-страницы, таблицы стилей CSS и JavaScript-модули, формирующие пользовательские интерфейсы и выполняющие обращения к API. На серверной стороне используется PHP-приложение, взаимодействующее с PostgreSQL через PDO.

Взаимодействие компонентов осуществляется по схеме:
\begin{enumerate}
\item пользователь инициирует действие в веб-интерфейсе;
\item JavaScript-модуль формирует HTTP-запрос к API;
\item серверный PHP-скрипт валидирует входные данные и права доступа;
\item при успешной проверке выполняется операция с БД;
\item результат возвращается в формате JSON и отображается в интерфейсе.
\end{enumerate}

\subsection{Модульная структура клиентской части}

Клиентская часть построена в виде набора специализированных модулей:
\begin{itemize}
\item \texttt{main.js} --- инициализация страницы, проверка авторизации, загрузка общих компонентов;
\item \texttt{sidebar.js} --- динамическая загрузка бокового меню в зависимости от роли;
\item \texttt{dashboard-*.js} --- логика личных кабинетов для администратора, организатора, школы, жюри и участника;
\item \texttt{applications-*.js}, \texttt{olympiad-detail.js} --- обработка заявок и детальных данных олимпиад;
\item \texttt{news-editor.js} --- управление новостным контентом.
\end{itemize}

Такой подход упрощает сопровождение проекта и изоляцию функциональных изменений.

\subsection{Серверная часть и API}

Серверная часть включает:
\begin{itemize}
\item модуль конфигурации подключения к БД (\texttt{config.php});
\item модуль авторизации и управления сессией (\texttt{login.php}, \texttt{check-auth.php}, \texttt{logout.php});
\item набор прикладных API в каталоге \texttt{api}.
\end{itemize}

API сгруппированы по подсистемам:
\begin{itemize}
\item управление олимпиадами: \texttt{create-olympiad.php}, \texttt{update-olympiad.php}, \texttt{get-olympiads.php};
\item управление участниками: \texttt{add-student.php}, \texttt{get-olympiad-participants.php}, \texttt{update-participant-score.php};
\item работа жюри: \texttt{get-jury-olympiads.php}, \texttt{publish-jury-results.php};
\item апелляции: \texttt{submit-appeal.php}, \texttt{respond-appeal.php};
\item сопровождение и коммуникации: \texttt{send-support-message.php}, \texttt{get-support-messages.php}.
\end{itemize}

\subsection{Проектирование бизнес-процесса «Проведение олимпиады»}

Ключевой бизнес-процесс можно представить последовательностью:
\begin{enumerate}
\item организатор создаёт олимпиаду и задаёт параметры (предмет, окно проведения, классы);
\item школа выбирает олимпиаду и формирует список участников;
\item участники загружают выполненные работы;
\item жюри оценивает работы и выставляет баллы;
\item публикуются результаты, участники при необходимости подают апелляции;
\item организатор и жюри рассматривают обращения и фиксируют итоговые решения.
\end{enumerate}

Данная схема покрывает полный цикл проведения школьного этапа и обеспечивает контроль статусов на каждом шаге.

\subsection{Описание ролевой модели доступа}

Ролевая модель реализована в таблице пользователей и проверяется серверной логикой при каждом критичном запросе. Основные права представлены в таблице~\ref{tab:roles}.

\begin{xltabular}{\textwidth}{|l|X|}
\caption{Распределение прав по ролям\label{tab:roles}}\\ \hline
\centrow Роль & \centrow Основные полномочия \\ \hline
\thead{1} & \thead{2} \\ \hline
\endfirsthead
\continuecaption{Продолжение таблицы \ref{tab:roles}}
\thead{1} & \thead{2} \\ \hline
\finishhead
Администратор & Управление пользователями, модерация, контроль системных операций \\ \hline
Организатор & Создание олимпиад, назначение жюри, публикация итогов \\ \hline
Школа/координатор & Регистрация участников, сопровождение заявок, просмотр результатов \\ \hline
Жюри & Проверка работ, выставление баллов, обработка апелляций \\ \hline
Ученик & Участие в олимпиадах, загрузка работ, подача апелляций
\end{xltabular}

\subsection{Нефункциональные характеристики}

При разработке и сопровождении проекта учитываются следующие нефункциональные требования:
\begin{itemize}
\item \textbf{Масштабируемость} --- возможность расширения API и добавления новых ролей;
\item \textbf{Сопровождаемость} --- модульная структура JavaScript и декомпозиция API по функциям;
\item \textbf{Производительность} --- выполнение операций доступа к данным через параметризованные запросы;
\item \textbf{Безопасность} --- контроль сессий, разграничение доступа и валидация входных данных.
\end{itemize}

\subsection{Результаты проектирования}

В рамках технического проекта сформирована целостная модель системы УДПСО, включающая архитектуру, ролевое разграничение и описание ключевых бизнес-процессов. Полученные результаты являются основой для дальнейшей доработки программной реализации и подготовки выпускной квалификационной работы.
