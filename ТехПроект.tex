\section{Технический проект}

\subsection{Общие проектные положения}

Технический проект определяет архитектурные и организационно-технологические решения для реализации программно-информационной системы проведения школьных предметных олимпиад. Проектные решения ориентированы на модульность, масштабируемость и соответствие нормативным требованиям к обработке персональных данных.

\subsection{Архитектура системы}

Система реализуется в виде трехуровневой архитектуры:
\begin{itemize}
\item \textbf{уровень представления} --- пользовательские веб-интерфейсы для различных ролей;
\item \textbf{уровень прикладной логики} --- серверные модули обработки запросов, проверки полномочий и выполнения бизнес-правил;
\item \textbf{уровень данных} --- централизованное хранение информации в реляционной базе данных.
\end{itemize}

Данный подход позволяет изолировать бизнес-логику от представления, упростить сопровождение и обеспечить развитие системы без радикальной переработки архитектурного каркаса.

\begin{figure}[H]
\centering
\fbox{\parbox[c][0.25\textheight][c]{0.93\linewidth}{\centering\textbf{ТРЕБУЕМОЕ ИЗОБРАЖЕНИЕ №3: АРХИТЕКТУРНАЯ СХЕМА (3 УРОВНЯ)}\\[1ex]
\textit{Нотация: прямоугольники и стрелки. Блоки: интерфейс, серверная логика, база данных}}}
\caption{Трехуровневая архитектура программно-информационной системы}
\label{fig:architecture-levels}
\end{figure}

\subsection{Функциональная декомпозиция}

Система включает следующие функциональные подсистемы:
\begin{enumerate}
\item подсистема управления пользователями и ролями;
\item подсистема управления олимпиадами и этапами проведения;
\item подсистема учета участников и регистрационных данных;
\item подсистема хранения и проверки работ;
\item подсистема публикации результатов и апелляций;
\item подсистема новостей, уведомлений и поддержки.
\end{enumerate}

Функциональные подсистемы проектируются как слабо связанные модули с четко определенными зонами ответственности.

\begin{figure}[H]
\centering
\fbox{\parbox[c][0.25\textheight][c]{0.93\linewidth}{\centering\textbf{ТРЕБУЕМОЕ ИЗОБРАЖЕНИЕ №4: ДЕКОМПОЗИЦИЯ ПОДСИСТЕМ}\\[1ex]
\textit{Нотация: иерархическая диаграмма. Корень: «Система олимпиад», ветви: 6 подсистем}}}
\caption{Декомпозиция функциональных подсистем}
\label{fig:functional-decomposition}
\end{figure}

\subsection{Информационная модель}

Информационная модель отражает основные сущности предметной области и связи между ними. Базовые сущности:
\begin{itemize}
\item пользователь;
\item олимпиада;
\item участник;
\item работа участника;
\item результат проверки;
\item апелляция.
\end{itemize}

Ключевые связи формируются по принципу «один-ко-многим» между олимпиадой и участниками, участником и работами, а также между результатами и апелляциями.

\begin{figure}[H]
\centering
\fbox{\parbox[c][0.25\textheight][c]{0.93\linewidth}{\centering\textbf{ТРЕБУЕМОЕ ИЗОБРАЖЕНИЕ №5: ER-ДИАГРАММА ДАННЫХ}\\[1ex]
\textit{Нотация: Crow's Foot. Обязательные сущности: пользователь, олимпиада, участник, работа, результат, апелляция}}}
\caption{Информационная модель данных системы}
\label{fig:er-model}
\end{figure}

\subsection{Модель бизнес-процесса}

Сквозной бизнес-процесс «Проведение олимпиады» включает этапы:
\begin{enumerate}
\item формирование параметров и запуск олимпиады;
\item регистрация участников и подтверждение заявок;
\item выполнение заданий и прием работ;
\item экспертная проверка и расчет результатов;
\item публикация итогов и обработка апелляций;
\item формирование итоговых протоколов.
\end{enumerate}

\begin{figure}[H]
\centering
\fbox{\parbox[c][0.25\textheight][c]{0.93\linewidth}{\centering\textbf{ТРЕБУЕМОЕ ИЗОБРАЖЕНИЕ №6: BPMN-СХЕМА ПРОЦЕССА ОЛИМПИАДЫ}\\[1ex]
\textit{Пулы: организатор, школа, жюри, участник. События: старт, публикация итогов}}}
\caption{Сквозной бизнес-процесс проведения олимпиады}
\label{fig:business-process-model}
\end{figure}

\subsection{Ролевая модель доступа}

Ролевое разграничение доступа является основой безопасной эксплуатации системы. Распределение полномочий по ролям представлено в таблице~\ref{tab:roles}.

\begin{xltabular}{\textwidth}{|l|X|}
\caption{Распределение полномочий пользователей\label{tab:roles}}\\ \hline
\centrow Роль & \centrow Основные полномочия \\ \hline
\thead{1} & \thead{2} \\ \hline
\endfirsthead
\continuecaption{Продолжение таблицы \ref{tab:roles}}
\thead{1} & \thead{2} \\ \hline
\finishhead
Администратор & Управление пользователями, настройками и контролем корректности данных \\ \hline
Организатор & Планирование олимпиад, управление этапами и публикация итогов \\ \hline
Представитель образовательной организации & Регистрация участников, сопровождение документов и мониторинг статусов \\ \hline
Член жюри & Проверка работ, выставление баллов, подготовка решений по апелляциям \\ \hline
Участник & Подача материалов, получение результатов, инициирование апелляции
\end{xltabular}

\subsection{Алгоритмическое обеспечение}

Для обеспечения корректной работы системы выделяются ключевые алгоритмические контуры:
\begin{itemize}
\item алгоритм проверки прав доступа при выполнении операций;
\item алгоритм валидации входных данных при регистрации и подаче материалов;
\item алгоритм публикации результатов с учетом статусов проверки;
\item алгоритм обработки апелляции с фиксацией принятых решений.
\end{itemize}

\begin{figure}[H]
\centering
\fbox{\parbox[c][0.25\textheight][c]{0.93\linewidth}{\centering\textbf{ТРЕБУЕМОЕ ИЗОБРАЖЕНИЕ №7: БЛОК-СХЕМА ОБРАБОТКИ АПЕЛЛЯЦИИ}\\[1ex]
\textit{Нотация: ГОСТ 19.701 или классическая блок-схема с развилками «принять/отклонить»}}}
\caption{Алгоритм обработки апелляционного обращения}
\label{fig:appeal-algorithm}
\end{figure}

\subsection{Рекомендации по подготовке иллюстраций (для ручной отрисовки)}

Для ускорения ручной подготовки графического материала рекомендуется формировать схемы в \textit{draw.io} с последующим экспортом в формат PDF/EPS (для схем) или PNG (для интерфейсных снимков). При оформлении следует придерживаться единых правил:
\begin{itemize}
\item единый шрифт и размер надписей на всех рисунках;
\item одинаковая толщина линий и стрелок внутри одного типа схем;
\item краткие подписи внутри блоков, подробности --- в тексте раздела;
\item нумерация рисунков автоматически средствами LaTeX;
\item подпись вида «Рисунок N -- Наименование» размещается под рисунком.
\end{itemize}

Состав обязательных иллюстраций для текущего отчёта представлен в таблице~\ref{tab:images-plan}.

\begin{xltabular}{\textwidth}{|C{0.9cm}|X|X|}
\caption{План подготовки иллюстраций\label{tab:images-plan}}\\ \hline
\centrow № & \centrow Что должно быть на изображении & \centrow Рекомендации по оформлению \\ \hline
\thead{1} & \thead{2} & \thead{3} \\ \hline
\endfirsthead
\continuecaption{Продолжение таблицы \ref{tab:images-plan}}
\thead{1} & \thead{2} & \thead{3} \\ \hline
\finishhead
1 & Организационная схема этапов олимпиады & Линейная блок-схема из 6--7 блоков, стрелки слева направо \\ \hline
2 & Схема взаимодействия ролей & Диаграмма прецедентов или сетевой граф ролей с подписями потоков \\ \hline
3 & Трехуровневая архитектура & Три горизонтальных слоя: UI, серверная логика, БД \\ \hline
4 & Декомпозиция подсистем & Иерархическая диаграмма: корневой блок и дочерние подсистемы \\ \hline
5 & ER-диаграмма & Сущности и связи 1:N, ключевые атрибуты в сокращенном виде \\ \hline
6 & BPMN-процесс & Пулы/дорожки по ролям, события старта/завершения \\ \hline
7 & Блок-схема апелляции & Ветвление по решению жюри, фиксация итогового статуса \\ \hline
8 & Схема модулей API проекта \texttt{udpso/api} & Группировка эндпоинтов по доменам: олимпиады, пользователи, апелляции, поддержка \\ \hline
9 & Пользовательский путь участника & Вход, выбор олимпиады, загрузка работы, просмотр результата, подача апелляции \\ \hline
10 & Пользовательский путь организатора & Создание олимпиады, назначение жюри, контроль проверки, публикация результатов \\ \hline
11 & Жизненный цикл статусов олимпиады & Диаграмма состояний: черновик, регистрация, проверка, публикация, архив \\ \hline
12 & Карта экранов личных кабинетов & Связи между страницами dashboard и рабочими разделами по ролям
\end{xltabular}

\subsection{Текстовые задания и промпты для генерации изображений}

Для всех иллюстраций рекомендуется единый стиль: светлый фон, деловая сине-серая палитра, читаемый шрифт без засечек, аккуратные стрелки и подписи на русском языке.

\begin{enumerate}
\item \textbf{Организационная схема этапов олимпиады.}\\
Что изобразить: последовательность этапов от подготовки олимпиады до публикации итогов.\\
Промпт: <<Создай профессиональную блок-схему на русском языке, показывающую этапы школьной предметной олимпиады: планирование, регистрация участников, проведение тура, загрузка работ, проверка жюри, апелляция, публикация итогов. Формат 16:9, светлый фон, синие блоки, серые стрелки слева направо, минималистичный стиль, без лишнего декора>>.

\item \textbf{Схема взаимодействия ролей системы.}\\
Что изобразить: роли и основные действия в системе.\\
Промпт: <<Создай UML use-case диаграмму на русском языке для системы проведения олимпиад. Акторы: Администратор, Организатор, Представитель школы, Член жюри, Участник. Прецеденты: управление олимпиадами, регистрация участников, проверка работ, публикация результатов, подача апелляции. Чистый академический стиль, белый фон, чёткие подписи>>.

\item \textbf{Трехуровневая архитектура системы.}\\
Что изобразить: слои представления, логики и данных.\\
Промпт: <<Нарисуй архитектурную схему software-системы в 3 слоя: уровень представления (веб-интерфейсы по ролям), уровень прикладной логики (REST API, авторизация, бизнес-правила), уровень данных (PostgreSQL, файлы работ, резервные копии). Добавь направленные стрелки между слоями. Стиль: техническая инфографика, строгий дизайн>>.

\item \textbf{Декомпозиция подсистем.}\\
Что изобразить: иерархию основных подсистем проекта.\\
Промпт: <<Создай иерархическую диаграмму функциональной декомпозиции системы олимпиад. Корневой блок: Программно-информационная система олимпиад. Ветви: пользователи и роли, управление олимпиадами, участники и заявки, работы и проверка, результаты и апелляции, новости и поддержка. Оформление в стиле инженерной схемы>>.

\item \textbf{ER-диаграмма данных.}\\
Что изобразить: сущности БД и связи между ними.\\
Промпт: <<Построй ER-диаграмму в нотации Crow's Foot на русском языке. Сущности: Пользователь, Роль, Олимпиада, Участник, Работа, Результат, Апелляция, Школа, Новость, Сообщение поддержки. Покажи ключевые связи 1:N и M:N через промежуточные таблицы, выдели первичные ключи. Формат для дипломного отчета, чистая читаемая схема>>.

\item \textbf{BPMN-процесс проведения олимпиады.}\\
Что изобразить: процесс по дорожкам ролей.\\
Промпт: <<Создай BPMN-диаграмму процесса проведения школьной олимпиады. Дорожки: Организатор, Школа, Участник, Жюри. События: старт процесса, окончание регистрации, завершение проверки, публикация результатов, завершение апелляций. Добавь gateways для ветвлений. Русский язык, строгий учебный стиль>>.

\item \textbf{Блок-схема обработки апелляции.}\\
Что изобразить: алгоритм от подачи до уведомления о решении.\\
Промпт: <<Нарисуй блок-схему алгоритма обработки апелляции участника. Шаги: прием апелляции, проверка сроков, передача жюри, рассмотрение, решение (удовлетворить или отклонить), фиксация результата, уведомление участника. Добавь условные ромбы и стрелки. Стиль классической инженерной блок-схемы>>.

\item \textbf{Схема модулей API проекта \texttt{udpso/api}.}\\
Что изобразить: логические группы API-эндпоинтов.\\
Промпт: <<Создай модульную схему REST API платформы олимпиад. Группы эндпоинтов: auth, users, olympiads, participants, results, appeals, support, files, news, dashboard stats. Покажи связь с frontend и базой данных. Современная техническая инфографика для дипломной работы, подписи на русском>>.

\item \textbf{Пользовательский путь участника.}\\
Что изобразить: типовой сценарий работы участника в системе.\\
Промпт: <<Создай customer journey map участника олимпиады на русском языке. Этапы: регистрация или вход, выбор олимпиады, подача заявки, загрузка работы, ожидание проверки, просмотр результата, подача апелляции. Для каждого этапа укажи цель пользователя и результат. Минималистичный академический стиль>>.

\item \textbf{Пользовательский путь организатора.}\\
Что изобразить: последовательность действий организатора от запуска до завершения олимпиады.\\
Промпт: <<Создай процессную диаграмму пути организатора олимпиады: создание олимпиады, настройка сроков, управление школами, назначение жюри, мониторинг статусов, публикация результатов, архивирование. Добавь контрольные точки и выходные документы. Русский язык, деловая схема для технического отчета>>.

\item \textbf{Схема жизненного цикла статусов олимпиады.}\\
Что изобразить: переходы между состояниями олимпиады.\\
Промпт: <<Нарисуй state machine диаграмму статусов олимпиады: Черновик, Открыта регистрация, Регистрация закрыта, Идет проверка, Результаты опубликованы, Апелляции завершены, Архив. Укажи условия переходов между состояниями. Строгий технический стиль без декоративных элементов>>.

\item \textbf{Карта экранов личных кабинетов.}\\
Что изобразить: структуру экранов и связи навигации по ролям.\\
Промпт: <<Создай карту экранов веб-платформы олимпиад с разделением по ролям: dashboard-admin, dashboard-organizer, dashboard-school, dashboard-jury, dashboard-student, страницы olympiads, applications, support, settings. Покажи навигационные переходы стрелками. Формат 16:9, wireframe-style, подписи на русском>>.
\end{enumerate}


\subsection{Диаграммы прецедентов по ролям и варианты использования}

Для каждой роли проекта \texttt{udpso} должна быть подготовлена отдельная диаграмма прецедентов UML. В отчете ниже фиксируются состав прецедентов и варианты использования, которые необходимо отразить на диаграммах и в текстовом описании.

\subsubsection{Администратор}

\begin{figure}[H]
\centering
\fbox{\parbox[c][0.23\textheight][c]{0.93\linewidth}{\centering\textbf{ТРЕБУЕМОЕ ИЗОБРАЖЕНИЕ №13: UML-ДИАГРАММА ПРЕЦЕДЕНТОВ ДЛЯ АДМИНИСТРАТОРА}\\[1ex]
\textit{Актор: Администратор. Прецеденты: управление пользователями, управление новостями, просмотр статистики, модерация системы}}}
\caption{Диаграмма прецедентов роли «Администратор»}
\label{fig:usecase-admin}
\end{figure}

\textbf{Варианты использования роли «Администратор»:}
\begin{enumerate}
\item \textbf{UC-ADM-01 Управление учетными записями.} Предусловие: администратор авторизован. Основной сценарий: просмотр списка пользователей, фильтрация, редактирование данных, сохранение изменений. Результат: обновленные сведения о пользователях зафиксированы в БД.
\item \textbf{UC-ADM-02 Назначение ролей.} Предусловие: выбран пользователь. Основной сценарий: выбор роли, проверка допустимости назначения, подтверждение операции. Результат: пользователю присвоена новая роль и права доступа.
\item \textbf{UC-ADM-03 Управление новостями.} Предусловие: администратор имеет доступ к модулю новостей. Основной сценарий: создание/редактирование публикации, прикрепление файла, публикация. Результат: новость отображается в публичном разделе.
\item \textbf{UC-ADM-04 Просмотр статистики платформы.} Предусловие: доступны агрегированные данные. Основной сценарий: открытие dashboard, выбор периода, анализ показателей. Результат: получены актуальные метрики по работе системы.
\end{enumerate}

\subsubsection{Организатор}

\begin{figure}[H]
\centering
\fbox{\parbox[c][0.23\textheight][c]{0.93\linewidth}{\centering\textbf{ТРЕБУЕМОЕ ИЗОБРАЖЕНИЕ №14: UML-ДИАГРАММА ПРЕЦЕДЕНТОВ ДЛЯ ОРГАНИЗАТОРА}\\[1ex]
\textit{Актор: Организатор. Прецеденты: создание олимпиады, настройка сроков, назначение жюри, публикация результатов, архивирование}}}
\caption{Диаграмма прецедентов роли «Организатор»}
\label{fig:usecase-organizer}
\end{figure}

\textbf{Варианты использования роли «Организатор»:}
\begin{enumerate}
\item \textbf{UC-ORG-01 Создание олимпиады.} Предусловие: организатор авторизован. Сценарий: ввод параметров олимпиады, сохранение карточки, установка дат. Результат: создана новая олимпиада в статусе подготовки.
\item \textbf{UC-ORG-02 Управление составом жюри.} Предусловие: олимпиада создана. Сценарий: добавление/удаление членов жюри, сохранение списка. Результат: утвержден актуальный состав проверяющих.
\item \textbf{UC-ORG-03 Работа с заявками школ.} Предусловие: открыт период регистрации. Сценарий: просмотр заявок, подтверждение или отклонение. Результат: сформирован окончательный перечень участников.
\item \textbf{UC-ORG-04 Публикация итогов.} Предусловие: проверка завершена. Сценарий: контроль итоговых баллов, запуск публикации, перевод в финальный статус. Результат: результаты доступны всем заинтересованным ролям.
\end{enumerate}

\subsubsection{Представитель образовательной организации}

\begin{figure}[H]
\centering
\fbox{\parbox[c][0.23\textheight][c]{0.93\linewidth}{\centering\textbf{ТРЕБУЕМОЕ ИЗОБРАЖЕНИЕ №15: UML-ДИАГРАММА ПРЕЦЕДЕНТОВ ДЛЯ ПРЕДСТАВИТЕЛЯ ШКОЛЫ}\\[1ex]
\textit{Актор: Представитель школы. Прецеденты: выбор олимпиады, регистрация участников, загрузка работ, просмотр статусов}}}
\caption{Диаграмма прецедентов роли «Представитель образовательной организации»}
\label{fig:usecase-school}
\end{figure}

\textbf{Варианты использования роли «Представитель образовательной организации»:}
\begin{enumerate}
\item \textbf{UC-SCH-01 Подача заявки школы на олимпиаду.} Предусловие: олимпиада доступна для регистрации. Сценарий: выбор олимпиады, заполнение формы, отправка заявки. Результат: заявка передана организатору.
\item \textbf{UC-SCH-02 Регистрация участников.} Предусловие: заявка школы подтверждена. Сценарий: добавление учеников, проверка корректности данных, сохранение. Результат: участники включены в список допущенных.
\item \textbf{UC-SCH-03 Загрузка работы участника.} Предусловие: открыт этап приема работ. Сценарий: выбор участника, загрузка файла, контроль формата. Результат: работа сохранена и доступна жюри.
\item \textbf{UC-SCH-04 Мониторинг результатов.} Предусловие: результаты опубликованы. Сценарий: просмотр баллов и протоколов, скачивание документов. Результат: школа получает официальные итоги.
\end{enumerate}

\subsubsection{Член жюри}

\begin{figure}[H]
\centering
\fbox{\parbox[c][0.23\textheight][c]{0.93\linewidth}{\centering\textbf{ТРЕБУЕМОЕ ИЗОБРАЖЕНИЕ №16: UML-ДИАГРАММА ПРЕЦЕДЕНТОВ ДЛЯ ЧЛЕНА ЖЮРИ}\\[1ex]
\textit{Актор: Член жюри. Прецеденты: проверка работ, выставление баллов, обработка апелляций, утверждение результатов}}}
\caption{Диаграмма прецедентов роли «Член жюри»}
\label{fig:usecase-jury}
\end{figure}

\textbf{Варианты использования роли «Член жюри»:}
\begin{enumerate}
\item \textbf{UC-JURY-01 Проверка работы участника.} Предусловие: работа загружена в систему. Сценарий: открытие работы, анализ, выставление баллов и комментариев. Результат: оценка сохранена.
\item \textbf{UC-JURY-02 Корректировка оценки.} Предусловие: публикация результатов еще не выполнена. Сценарий: повторный просмотр, изменение баллов, подтверждение. Результат: обновлена итоговая оценка.
\item \textbf{UC-JURY-03 Рассмотрение апелляции.} Предусловие: от участника поступила апелляция в срок. Сценарий: изучение обращения, принятие решения, фиксация обоснования. Результат: апелляция закрыта с официальным решением.
\item \textbf{UC-JURY-04 Подготовка итоговых протоколов.} Предусловие: завершена проверка всех работ. Сценарий: сверка баллов, формирование итоговых таблиц, передача на публикацию. Результат: подготовлен комплект итоговых данных.
\end{enumerate}

\subsubsection{Участник олимпиады}

\begin{figure}[H]
\centering
\fbox{\parbox[c][0.23\textheight][c]{0.93\linewidth}{\centering\textbf{ТРЕБУЕМОЕ ИЗОБРАЖЕНИЕ №17: UML-ДИАГРАММА ПРЕЦЕДЕНТОВ ДЛЯ УЧАСТНИКА}\\[1ex]
\textit{Актор: Участник. Прецеденты: регистрация, подача заявки, загрузка работы, просмотр результатов, подача апелляции}}}
\caption{Диаграмма прецедентов роли «Участник»}
\label{fig:usecase-student}
\end{figure}

\textbf{Варианты использования роли «Участник»:}
\begin{enumerate}
\item \textbf{UC-STU-01 Регистрация и вход в систему.} Предусловие: пользователь не имеет активной сессии. Сценарий: ввод регистрационных данных, подтверждение почты, авторизация. Результат: создана учетная запись и открыт личный кабинет.
\item \textbf{UC-STU-02 Выбор олимпиады и участие.} Предусловие: опубликован список доступных олимпиад. Сценарий: просмотр условий, подача заявки, ожидание подтверждения. Результат: участник допущен к олимпиаде.
\item \textbf{UC-STU-03 Загрузка конкурсной работы.} Предусловие: прием работ открыт. Сценарий: прикрепление файла, проверка формата, отправка. Результат: работа зарегистрирована в системе.
\item \textbf{UC-STU-04 Получение результата и апелляция.} Предусловие: результаты опубликованы. Сценарий: просмотр баллов, при несогласии --- формирование апелляции. Результат: участник получает официальный ответ по итогам апелляции.
\end{enumerate}

\subsection{Промпты для генерации диаграмм прецедентов по ролям}

\begin{enumerate}
\item \textbf{Администратор.}\\
Промпт: <<Создай UML use-case диаграмму на русском языке для роли Администратор в системе школьных олимпиад. Актор: Администратор. Прецеденты: управление пользователями, назначение ролей, управление новостями, просмотр статистики, модерация обращений. Стиль строгий, белый фон, черно-синие элементы, подписи четкие>>.

\item \textbf{Организатор.}\\
Промпт: <<Создай UML use-case диаграмму на русском языке для роли Организатор. Прецеденты: создание олимпиады, настройка параметров, назначение жюри, управление заявками школ, публикация результатов, архивирование олимпиады. Академический стиль для диплома, без декоративных элементов>>.

\item \textbf{Представитель школы.}\\
Промпт: <<Создай UML use-case диаграмму для роли Представитель образовательной организации. Прецеденты: выбор олимпиады, подача заявки школы, регистрация участников, загрузка работ, просмотр статусов и результатов. Подписи на русском, аккуратная техническая графика>>.

\item \textbf{Член жюри.}\\
Промпт: <<Создай UML use-case диаграмму для роли Член жюри. Прецеденты: просмотр назначенных работ, выставление баллов, корректировка оценок, рассмотрение апелляций, подготовка итоговых протоколов. Строгий стиль, формат для отчета ВКР>>.

\item \textbf{Участник.}\\
Промпт: <<Создай UML use-case диаграмму для роли Участник олимпиады. Прецеденты: регистрация и вход, выбор олимпиады, подача заявки, загрузка работы, просмотр результатов, подача апелляции. Минималистичная понятная схема на русском языке>>.
\end{enumerate}

\subsection{Требования к эксплуатационным характеристикам}

Проектом предусматриваются следующие эксплуатационные характеристики:
\begin{itemize}
\item \textbf{надежность} --- устойчивость выполнения операций при типовых сценариях нагрузки;
\item \textbf{безопасность} --- ограничение доступа к данным в соответствии с ролями;
\item \textbf{сопровождаемость} --- возможность расширения функционала без нарушения архитектурной целостности;
\item \textbf{масштабируемость} --- возможность увеличения количества пользователей и олимпиадных мероприятий.
\end{itemize}

\subsection{Выводы по разделу}

Представленные проектные решения формируют техническую основу разработки программно-информационной системы для проведения школьных предметных олимпиад. Архитектурные и функциональные решения обеспечивают возможность поэтапного развития системы и соответствуют целям преддипломной практики и теме ВКР.
