\section{Технический проект}

\subsection{Общие проектные положения}

Технический проект определяет архитектурные и организационно-технологические решения для реализации программно-информационной системы проведения школьных предметных олимпиад. Проектные решения ориентированы на модульность, масштабируемость и соответствие нормативным требованиям к обработке персональных данных.

\subsection{Архитектура системы}

Система реализуется в виде трехуровневой архитектуры:
\begin{itemize}
\item \textbf{уровень представления} --- пользовательские веб-интерфейсы для различных ролей;
\item \textbf{уровень прикладной логики} --- серверные модули обработки запросов, проверки полномочий и выполнения бизнес-правил;
\item \textbf{уровень данных} --- централизованное хранение информации в реляционной базе данных.
\end{itemize}

Данный подход позволяет изолировать бизнес-логику от представления, упростить сопровождение и обеспечить развитие системы без радикальной переработки архитектурного каркаса.

\begin{figure}[H]
\centering
\fbox{\parbox[c][0.25\textheight][c]{0.93\linewidth}{\centering\textbf{АРХИТЕКТУРНАЯ СХЕМА (УРОВНИ ПРЕДСТАВЛЕНИЯ, ЛОГИКИ И ДАННЫХ)}\\[1ex]
\textit{МЕСТО ДЛЯ ВСТАВКИ ИЗОБРАЖЕНИЯ}}}
\caption{Трехуровневая архитектура программно-информационной системы}
\label{fig:architecture-levels}
\end{figure}

\subsection{Функциональная декомпозиция}

Система включает следующие функциональные подсистемы:
\begin{enumerate}
\item подсистема управления пользователями и ролями;
\item подсистема управления олимпиадами и этапами проведения;
\item подсистема учета участников и регистрационных данных;
\item подсистема хранения и проверки работ;
\item подсистема публикации результатов и апелляций;
\item подсистема новостей, уведомлений и поддержки.
\end{enumerate}

Функциональные подсистемы проектируются как слабо связанные модули с четко определенными зонами ответственности.

\begin{figure}[H]
\centering
\fbox{\parbox[c][0.25\textheight][c]{0.93\linewidth}{\centering\textbf{ФУНКЦИОНАЛЬНАЯ ДЕКОМПОЗИЦИЯ СИСТЕМЫ}\\[1ex]
\textit{МЕСТО ДЛЯ ВСТАВКИ ИЗОБРАЖЕНИЯ}}}
\caption{Декомпозиция функциональных подсистем}
\label{fig:functional-decomposition}
\end{figure}

\subsection{Информационная модель}

Информационная модель отражает основные сущности предметной области и связи между ними. Базовые сущности:
\begin{itemize}
\item пользователь;
\item олимпиада;
\item участник;
\item работа участника;
\item результат проверки;
\item апелляция.
\end{itemize}

Ключевые связи формируются по принципу «один-ко-многим» между олимпиадой и участниками, участником и работами, а также между результатами и апелляциями.

\begin{figure}[H]
\centering
\fbox{\parbox[c][0.25\textheight][c]{0.93\linewidth}{\centering\textbf{ER-ДИАГРАММА ОСНОВНЫХ СУЩНОСТЕЙ И СВЯЗЕЙ}\\[1ex]
\textit{МЕСТО ДЛЯ ВСТАВКИ ИЗОБРАЖЕНИЯ}}}
\caption{Информационная модель данных системы}
\label{fig:er-model}
\end{figure}

\subsection{Модель бизнес-процесса}

Сквозной бизнес-процесс «Проведение олимпиады» включает этапы:
\begin{enumerate}
\item формирование параметров и запуск олимпиады;
\item регистрация участников и подтверждение заявок;
\item выполнение заданий и прием работ;
\item экспертная проверка и расчет результатов;
\item публикация итогов и обработка апелляций;
\item формирование итоговых протоколов.
\end{enumerate}

\begin{figure}[H]
\centering
\fbox{\parbox[c][0.25\textheight][c]{0.93\linewidth}{\centering\textbf{ДИАГРАММА БИЗНЕС-ПРОЦЕССА «ПРОВЕДЕНИЕ ОЛИМПИАДЫ»}\\[1ex]
\textit{МЕСТО ДЛЯ ВСТАВКИ ИЗОБРАЖЕНИЯ}}}
\caption{Сквозной бизнес-процесс проведения олимпиады}
\label{fig:business-process-model}
\end{figure}

\subsection{Ролевая модель доступа}

Ролевое разграничение доступа является основой безопасной эксплуатации системы. Распределение полномочий по ролям представлено в таблице~\ref{tab:roles}.

\begin{xltabular}{\textwidth}{|l|X|}
\caption{Распределение полномочий пользователей\label{tab:roles}}\\ \hline
\centrow Роль & \centrow Основные полномочия \\ \hline
\thead{1} & \thead{2} \\ \hline
\endfirsthead
\continuecaption{Продолжение таблицы \ref{tab:roles}}
\thead{1} & \thead{2} \\ \hline
\finishhead
Администратор & Управление пользователями, настройками и контролем корректности данных \\ \hline
Организатор & Планирование олимпиад, управление этапами и публикация итогов \\ \hline
Представитель образовательной организации & Регистрация участников, сопровождение документов и мониторинг статусов \\ \hline
Член жюри & Проверка работ, выставление баллов, подготовка решений по апелляциям \\ \hline
Участник & Подача материалов, получение результатов, инициирование апелляции
\end{xltabular}

\subsection{Алгоритмическое обеспечение}

Для обеспечения корректной работы системы выделяются ключевые алгоритмические контуры:
\begin{itemize}
\item алгоритм проверки прав доступа при выполнении операций;
\item алгоритм валидации входных данных при регистрации и подаче материалов;
\item алгоритм публикации результатов с учетом статусов проверки;
\item алгоритм обработки апелляции с фиксацией принятых решений.
\end{itemize}

\begin{figure}[H]
\centering
\fbox{\parbox[c][0.25\textheight][c]{0.93\linewidth}{\centering\textbf{БЛОК-СХЕМА АЛГОРИТМА ОБРАБОТКИ АПЕЛЛЯЦИИ}\\[1ex]
\textit{МЕСТО ДЛЯ ВСТАВКИ ИЗОБРАЖЕНИЯ}}}
\caption{Алгоритм обработки апелляционного обращения}
\label{fig:appeal-algorithm}
\end{figure}

\subsection{Требования к эксплуатационным характеристикам}

Проектом предусматриваются следующие эксплуатационные характеристики:
\begin{itemize}
\item \textbf{надежность} --- устойчивость выполнения операций при типовых сценариях нагрузки;
\item \textbf{безопасность} --- ограничение доступа к данным в соответствии с ролями;
\item \textbf{сопровождаемость} --- возможность расширения функционала без нарушения архитектурной целостности;
\item \textbf{масштабируемость} --- возможность увеличения количества пользователей и олимпиадных мероприятий.
\end{itemize}

\subsection{Выводы по разделу}

Представленные проектные решения формируют техническую основу разработки программно-информационной системы для проведения школьных предметных олимпиад. Архитектурные и функциональные решения обеспечивают возможность поэтапного развития системы и соответствуют целям преддипломной практики и теме ВКР.
