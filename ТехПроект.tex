\section{Технический проект}

\subsection{Архитектурное решение}

Программно-информационная система для проведения школьных предметных олимпиад реализуется по клиент--серверной архитектуре. Пользовательские интерфейсы обеспечивают взаимодействие с системой через веб-браузер, серверная часть выполняет обработку запросов, бизнес-логику и доступ к данным, а хранилище обеспечивает долговременное и целостное хранение информации.

Обмен данными между компонентами осуществляется по HTTP с передачей структурированных ответов в формате JSON.

\begin{figure}[H]
\centering
\fbox{\parbox[c][0.24\textheight][c]{0.92\linewidth}{\centering\textbf{СТРУКТУРНАЯ СХЕМА АРХИТЕКТУРЫ СИСТЕМЫ}\\[1ex]
\textit{МЕСТО ДЛЯ ВСТАВКИ ИЗОБРАЖЕНИЯ}}}
\caption{Архитектура программно-информационной системы}
\label{fig:architecture}
\end{figure}

\subsection{Функциональная структура}

Система включает следующие функциональные подсистемы:
\begin{itemize}
\item подсистема управления пользователями и правами доступа;
\item подсистема управления олимпиадами и календарём проведения;
\item подсистема учёта участников и работ;
\item подсистема проверки работ, публикации результатов и апелляций;
\item подсистема новостей, уведомлений и обратной связи.
\end{itemize}

Функциональная декомпозиция обеспечивает масштабируемость решения и возможность независимого развития модулей.

\begin{figure}[H]
\centering
\fbox{\parbox[c][0.24\textheight][c]{0.92\linewidth}{\centering\textbf{ДИАГРАММА ФУНКЦИОНАЛЬНЫХ ПОДСИСТЕМ}\\[1ex]
\textit{МЕСТО ДЛЯ ВСТАВКИ ИЗОБРАЖЕНИЯ}}}
\caption{Функциональная структура системы}
\label{fig:functional-structure}
\end{figure}

\subsection{Проектирование бизнес-процесса проведения олимпиады}

Ключевой процесс включает последовательность:
\begin{enumerate}
\item создание олимпиады и задание параметров проведения;
\item регистрация участников от образовательных организаций;
\item выполнение заданий и загрузка работ;
\item проверка работ членами жюри и выставление баллов;
\item публикация результатов и обработка апелляций;
\item формирование итоговой отчётности.
\end{enumerate}

\begin{figure}[H]
\centering
\fbox{\parbox[c][0.24\textheight][c]{0.92\linewidth}{\centering\textbf{СХЕМА БИЗНЕС-ПРОЦЕССА ПРОВЕДЕНИЯ ОЛИМПИАДЫ}\\[1ex]
\textit{МЕСТО ДЛЯ ВСТАВКИ ИЗОБРАЖЕНИЯ}}}
\caption{Бизнес-процесс проведения школьной предметной олимпиады}
\label{fig:business-process}
\end{figure}

\subsection{Ролевая модель доступа}

В системе применяется ролевая модель, в рамках которой каждой категории пользователей назначается набор разрешённых операций. Распределение полномочий представлено в таблице~\ref{tab:roles}.

\begin{xltabular}{\textwidth}{|l|X|}
\caption{Распределение прав доступа по ролям\label{tab:roles}}\\ \hline
\centrow Роль & \centrow Основные полномочия \\ \hline
\thead{1} & \thead{2} \\ \hline
\endfirsthead
\continuecaption{Продолжение таблицы \ref{tab:roles}}
\thead{1} & \thead{2} \\ \hline
\finishhead
Администратор & Управление пользователями, контроль корректности данных, сопровождение эксплуатации \\ \hline
Организатор & Планирование олимпиад, назначение жюри, публикация итогов \\ \hline
Представитель образовательной организации & Регистрация участников, сопровождение подачи работ и заявок \\ \hline
Жюри & Проверка работ, выставление оценок, участие в рассмотрении апелляций \\ \hline
Участник олимпиады & Доступ к заданиям, загрузка работы, просмотр результатов
\end{xltabular}

\subsection{Требования к качественным характеристикам}

Проектируемая система должна удовлетворять следующим нефункциональным требованиям:
\begin{itemize}
\item \textbf{надёжность} --- устойчивое выполнение основных операций и сохранность данных;
\item \textbf{безопасность} --- защита персональных данных и контроль доступа;
\item \textbf{сопровождаемость} --- возможность дальнейшего развития без изменения базовой архитектуры;
\item \textbf{производительность} --- приемлемое время отклика при одновременной работе нескольких категорий пользователей.
\end{itemize}

\subsection{Выводы по разделу}

Разработанные проектные решения определяют основу построения программно-информационной системы для проведения школьных предметных олимпиад и обеспечивают переход к этапам дальнейшей реализации и внедрения.
